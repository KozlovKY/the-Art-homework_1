\documentclass[12pt]{article}
\usepackage[utf8]{inputenc}
\usepackage[T1]{fontenc}
\usepackage{amsmath,amsfonts,amssymb}
\usepackage{graphicx}
\usepackage{a4wide}
\title{Reconstructed abstract of the paper Neural Ordinary Differential Equations}
%\author{not specified, not necessary here}
\date{}
\begin{document}
\maketitle

\begin{abstract}
This paper presents a new approach to deep learning that treats neural networks as dynamic systems with continuous time rather than the traditional discrete sequence of transformations (layers). The key idea is to parameterise the continuous
dynamics of hidden units using an ordinary differential equation (ODE) specified by neural network. A adjoint gradient method for end-to-end training is shown using continuous normalising flows as an example.

\end{abstract}
\paragraph{Keywords:} 

Continuous-time dynamics,
ODE solvers,
Adjoint method,
Time-series prediction,
Latent variable models





\paragraph{Highlights:}
\begin{enumerate}
\item Neural ODEs generalize neural networks by modeling transformations as continuous-time dynamical systems using ordinary differential equations
\item Neural ODEs use the adjoint method for memory-efficient backpropagation.
\item The model dynamically adjusts computation based on the complexity of the input.
\item Neural ODEs are well-suited for irregular time-series prediction, physics simulations, and latent variable modeling.
\end{enumerate}

\section{Introduction}
In a quantum mechanics problem one follows the evolution of the projection of the spin of a qubit over time, if one considers a system with two or more qubits, then by measuring the spin on one qubit the quantum state of the overall system is destroyed, and this is a problem, one wants to make an estimate on the spin of an individual particle without making a measurement.  This method seems attractive for recovering the hidden state dynamics of an individual qubit and predicting the spin value
\end{document}